\documentclass{article}
\usepackage[utf8]{inputenc}
\usepackage{geometry}
\geometry{textwidth=7cm}

\usepackage{amsmath}

\title{Display Style Example}
\author{Overleaf team}
\begin{document}

\maketitle

Depending on the value of $x$ the equation \( f(x) = \sum_{i=0}^{n} \frac{a_i}{1+x} \) may diverge or converge.

\[ f(x) = \sum_{i=0}^{n} \frac{a_i}{1+x} \]

\vspace{1cm}

In-line maths elements can be set with a different style: \(f(x) = \displaystyle \frac{1}{1+x}\). The same is true the other way around:

\begin{eqnarray*}
f(x) = \sum_{i=0}^{n} \frac{a_i}{1+x} \\
\textstyle f(x) = \sum_{i=0}^{n} \frac{a_i}{1+x} \\
\scriptstyle f(x) = \sum_{i=0}^{n} \frac{a_i}{1+x} \\
\scriptscriptstyle f(x) = \sum_{i=0}^{n} \frac{a_i}{1+x}
\end{eqnarray*}

\end{document}
